\documentclass{artikel3}

\usepackage{hyperref}
\usepackage{mathtools}
\usepackage{multicol}
\usepackage{esdiff}

\begin{document}

\title{Calculus BC Important Info Sheet}
\author{Transcribed to \LaTeX\ by Jack Ellert-Beck\\ 
Most information compiled by Dr.\ Olga Voronkova}
\maketitle

\newpage

\tableofcontents

\newpage

\part{Information}

\section{Functions}

\subsection{Even and Odd Functions}
A function $y=f(x)$ is \textbf{even} if $f(-x)=f(x)$ for every $x$ 
in the function's domain. Every even function is summertric about 
the $y$-axis. \[\text{Example:}\hspace{6 mm} (-x)^2=x^2 \]
A function $y=f(x)$ is \textbf{odd} if $f(x)=-f(x)$ for every $x$ 
in the function's domain. Every odd function is symmertric about 
the origin. \[\text{Example:}\hspace{6 mm} (-x)^3=-(x)^3 \]

\subsection{Periodicity}
A function $f(x)$ is \textbf{periodic} with period $p\hspace{2 mm}(p>0)$ if 
$f(x+p)=f(x)$ for every value of $x$.

Sinusoids are examples of periodic functions. Specifically, the period
of the function $y=A\sin(Bx+C)$ or $y=A\cos(Bx+C)$ is $\frac{2\pi}{|B|}$.
The amplitude is $|A|$. The period of $y=\tan(x)$ is $\pi$.

\subsection{Inverse Functions}
The inverse of a funtion $f(x)$ is often written as $f^{-1}(x)$, or given 
a new name such as $g(x)$.

If $f$ and $g$ are two functions such that $f(g(x))=x$ for every $x$ in 
the domain of $g$, and $g(f(x))=x$ for every $x$ in the domain of $f$, then
$f$ and $g$ are inverse functions of one another.

A function $f$ has an inverse if and only if no horizontal line intersects its graph more than once.

If $f$ is either increasing or decreasing in an interval, then $f$ has an inverse.

For information on the derivatives of inverse functions, see Section~\ref{dinv}.

\section{Limits}

\subsection{Definition of a Limit}
Let $f$ be a function defined on an open interval containing $c$ 
(except possibly at $c$) and let $L$ be a real number.
Then $\lim_{x \to c} f(x)=L$ means that for each $\varepsilon >0$
there exists a $\delta >0$ such that $|f(x)-L|<\varepsilon$ whenever
$0<|x-c|<\delta$.

\subsubsection{Symbolic Definitions of Limits}
\begin{itemize}
\item{Definition of a finite limit at a specific point:
\[ \lim_{x \to c}=L\Leftrightarrow (\forall\varepsilon >0, \exists\delta >0, 
0<|x-c|<\delta \Rightarrow |f(x)-L|<\varepsilon)\]}
\item{Definition of an undefined limit at a specific point:
\[ \lim_{x \to c}=\infty\Leftrightarrow (\forall M>0, \exists\delta >0,
0<|x-c|<\delta\Rightarrow f(x)>M)\]
\[ \lim_{x \to c}=-\infty\Leftrightarrow (\forall M<0, \exists\delta >0,
0<|x-c|<\delta\Rightarrow f(x)<M)\]}
\item{Definition of a finite limit at infinity:
\[ \lim_{x \to \infty}=L\Leftrightarrow (\forall\varepsilon>0,\exists M,
x>M\Rightarrow|f(x)-L|<\varepsilon) \]
\[ \lim_{x \to -\infty}=L\Leftrightarrow (\forall\varepsilon>0,\exists M,
x<M\Rightarrow|f(x)-L|<\varepsilon) \]}
\end{itemize}

\subsection{Continuity}
A function $y=f(x)$ is \textbf{continuous} at $x=a$ if $f(a)$ exists, 
$\displaystyle\lim_{x \to a}f(x)$ exists, and $\displaystyle\lim_{x \to a}f(x)=f(a)$.
$y=f(x)$ is continuous on $(a,b)$ if $f(x)$ is continuous for every
$x\in(a,b)$.

\subsection{Horizontal and Vertical Asymptotes}
A line $y=b$ is a \textbf{horizontal asymptote} of the graph of $y=f(x)$ if either\\*
$\displaystyle\lim_{x \to \infty}f(x)=b \text{ or} \lim_{x \to -\infty}f(x)=b$.

A line $x=a$ is a \textbf{vertical asymptote} of the graph of $y=f(x)$ if either\\*
$\displaystyle\lim_{x \to a^+}f(x)=\pm\infty \text{ or} \lim_{x \to a^-}f(x)=\pm\infty$.

\subsection{Evaluating Limits}

\subsubsection{Limits of Rational Functions as $x \to \pm\infty$}
\begin{itemize}
\item{$\displaystyle\lim_{x \to \pm\infty}\frac{f(x)}{g(x)}=0$ if the degree
of $f(x)$ is less than the degree of $g(x)$.
\[ \text{Example:}\hspace{6 mm}\lim_{x \to \infty}\frac{x^2-2x}{x^3+3}=0 \]}
\item{$\displaystyle\lim_{x \to \pm\infty}\frac{f(x)}{g(x)}$ is infinite if the degree
of $f(x)$ is greater than the degree of $g(x)$.
\[ \text{Example:}\hspace{6 mm}\lim_{x \to \infty}\frac{x^3+2x}{x^2-8}=\infty \]}
\item{$\displaystyle\lim_{x \to \pm\infty}\frac{f(x)}{g(x)}$ is finite if the degree 
of $f(x)$ is equal to the degree of $g(x)$. The limit will be equal to the ratio of 
the leading coefficients of $f(x)$ to $g(x)$.
\[ \text{Example:}\hspace{6 mm}\lim_{x \to \infty}\frac{2x^2-3x+2}{10x-5x^2}=-\frac{2}{5} \]}
\end{itemize}

\subsubsection{Remarkable Limits}
See THE THING THAT HAPPENS LATER.

\subsection{Intermediate Value Theorem}
A function $y=f(x)$ that is continuous on a closed interval $[a,b]$ takes on 
every value between $f(a)$ and $f(b)$.
As a more specific result, if $f$ is continuous on $[a,b]$ and $f(a)$ and $f(b)$ 
differ in sign, then the equation $f(x)=0$ has at least one solution 
in the open interval $[a,b]$.

\section{Derivatives}

\subsection{Notation of Derivatives}
The derivative of the function $y=f(x)$ is commonly written as any of the following:
\[ f'(x)\hspace{6 mm}[f(x)]'\hspace{6 mm}y'\]
\[ \diff{y}{x}\hspace{6 mm}\diff{}{x}f(x) \]

Higher order derivatives can be written in a number of ways. Below are the 
respective notations for second, fourth, and nth order derivatives in three different styles.
\[ \diff[2]{y}{x}\hspace{6 mm}\diff[4]{y}{x}\hspace{6 mm}\diff[n]{y}{x} \]
\[ f''(x)\hspace{6 mm}f^{IV}(x)\hspace{6 mm}f^{N}(x) \]
\[ f^{(2)}(x)\hspace{6 mm}f^{(4)}(x)\hspace{6 mm}f^{(n)} \]

\subsection{Rate of Change}
If $(x_0,y_0)$ and $(x_1,y_1)$ are points on the graph of $y=f(x)$, then 
the \textbf{average rate of change} of $y$ with respect to $x$ over 
the interval $[x_0,x_1]$ is
\[ \frac{f(x_1)-f(x_0)}{x_1-x_0}=\frac{y_1-y_0}{x_1-x_0}=\frac{\Delta y}{\Delta x} \]

If $(x_0,y_0)$ is a point on the graph of $y=f(x)$, then the \textbf{instantaneous 
rate of change} of $y$ with respect to $x$ at $x_0$ is $f'(x_0)$.

\subsection{Definition of a Derivative}
The derivative of a function $f(x)$ at the point $x=a$ can be defined in either of two ways:
\[ f'(a)=\lim_{h \to 0}\frac{f(a+h)-f(a)}{h} \]
\[ f'(a)=\lim_{b \to a}\frac{f(b)-f(a)}{b-a} \]
If this limit exists at $a$, then $f(x)$ is said to be \textbf{differentiable} at $a$.

If a function is differentiable at a point $x=a$, it is continuous at that point.
The converse is false, i.e.\ continuity does not imply differentiability.

\subsection{Derivatives of Inverse Functions}\label{dinv}
If $f$ is differentiable at every point on an interval $I$, and $f'(x)\neq 0$
on $I$, then $g=f^{-1}(x)$ is differentiable at every point of the interior of
the interval $f(I)$ and $g'(f(x))=\frac{1}{f'(x)}$.

\subsection{Finding Maxima and Minima}

\subsection{Monotonicity}

\subsection{Concavity}

\subsection{Rolle's Theorem}

\subsection{Mean Value Theorem}

\subsection{Extreme Value Theorem}

\subsection{Linear Approximation}

\subsection{Newton's Method}

\subsection{L'H\^{o}pital's Rule}

Lorem Ipsum

\end{document}
