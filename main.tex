\documentclass{artikel3}

\usepackage{hyperref}
\usepackage{mathtools}
\usepackage{multicol}
\usepackage{esdiff}

\hypersetup{
  colorlinks   = true, %Colours links instead of ugly boxes
  urlcolor     = blue, %Colour for external hyperlinks
  linkcolor    = red, %Colour of internal links
  citecolor   = red %Colour of citations
}

\newcommand{\ddx}{\diff{}{x}}

\begin{document}

\title{Calculus BC Important Info Sheet}
\author{Transcribed to \LaTeX\ by Jack Ellert-Beck\\ 
Most information compiled by Dr.\ Olga Voronkova}
\maketitle

\newpage

\tableofcontents

\newpage

\part{Information}

\section{Functions}

\subsection{Even and Odd Functions}
A function $y=f(x)$ is \textbf{even} if $f(-x)=f(x)$ for every $x$ 
in the function's domain. Every even function is summertric about 
the $y$-axis. 
\[\text{Example:}\hspace{6 mm} {(-x)}^2=x^2 \]
A function $y=f(x)$ is \textbf{odd} if $f(x)=-f(x)$ for every $x$ 
in the function's domain. Every odd function is symmertric about 
the origin. 
\[\text{Example:}\hspace{6 mm} {(-x)}^3=-{(x)}^3 \]

\subsection{Periodicity}
A function $f(x)$ is \textbf{periodic} with period $p\hspace{2 mm}(p>0)$ if 
$f(x+p)=f(x)$ for every value of $x$.

Sinusoids are examples of periodic functions. Specifically, the period
of the function $y=A\sin(Bx+C)$ or $y=A\cos(Bx+C)$ is $\frac{2\pi}{|B|}$.
The amplitude is $|A|$. The period of $y=\tan(x)$ is $\pi$.

\subsection{Composition of Functions}
Given two functions $f$ and $g$, the composite function
$f(g(x))$ can be written as $(f\circ g)(x)$. Note that 
$f\circ g$ is not necessarily equal to $g\circ f$.

\subsection{Inverse Functions}
The inverse of a funtion $f(x)$ is often written as $f^{-1}(x)$, or given 
a new name such as $g(x)$.

If $f$ and $g$ are two functions such that $f(g(x))=x$ for every $x$ in 
the domain of $g$, and $g(f(x))=x$ for every $x$ in the domain of $f$, then
$f$ and $g$ are inverse functions of one another.

A function $f$ has an inverse if and only if no horizontal line intersects its graph more than once.

If $f$ is either increasing or decreasing in an interval, then $f$ has an inverse.

For information on the derivatives of inverse functions, see Section~\ref{dinv}.

\section{Limits}

\subsection{Definition of a Limit}
Let $f$ be a function defined on an open interval containing $c$ 
(except possibly at $c$) and let $L$ be a real number.
Then $\lim_{x \to c} f(x)=L$ means that for each $\varepsilon >0$
there exists a $\delta >0$ such that $|f(x)-L|<\varepsilon$ whenever
$0<|x-c|<\delta$.

\subsubsection{Symbolic Definitions of Limits}
\begin{itemize}
\item{Definition of a finite limit at a specific point:
\[ \lim_{x \to c}=L\Leftrightarrow (\forall\varepsilon >0, \exists\delta >0, 
0<|x-c|<\delta \Rightarrow |f(x)-L|<\varepsilon)\]}
\item{Definition of an undefined limit at a specific point:
\[ \lim_{x \to c}=\infty\Leftrightarrow (\forall M>0, \exists\delta >0,
0<|x-c|<\delta\Rightarrow f(x)>M)\]
\[ \lim_{x \to c}=-\infty\Leftrightarrow (\forall M<0, \exists\delta >0,
0<|x-c|<\delta\Rightarrow f(x)<M)\]}
\item{Definition of a finite limit at infinity:
\[ \lim_{x \to \infty}=L\Leftrightarrow (\forall\varepsilon>0,\exists M,
x>M\Rightarrow|f(x)-L|<\varepsilon) \]
\[ \lim_{x \to -\infty}=L\Leftrightarrow (\forall\varepsilon>0,\exists M,
x<M\Rightarrow|f(x)-L|<\varepsilon) \]}
\end{itemize}

\subsection{Continuity}
A function $y=f(x)$ is \textbf{continuous} at $x=a$ if $f(a)$ exists, 
$\displaystyle\lim_{x \to a}f(x)$ exists, and $\displaystyle\lim_{x \to a}f(x)=f(a)$.
$y=f(x)$ is continuous on $(a,b)$ if $f(x)$ is continuous for every
$x\in(a,b)$.

\subsection{Horizontal and Vertical Asymptotes}
A line $y=b$ is a \textbf{horizontal asymptote} of the graph of $y=f(x)$ if either\\*
$\displaystyle\lim_{x \to \infty}f(x)=b \text{ or} \lim_{x \to -\infty}f(x)=b$.

A line $x=a$ is a \textbf{vertical asymptote} of the graph of $y=f(x)$ if either\\*
$\displaystyle\lim_{x \to a^+}f(x)=\pm\infty \text{ or} \lim_{x \to a^-}f(x)=\pm\infty$.

\subsection{Evaluating Limits}

\subsubsection{Limits of Rational Functions as $x \to \pm\infty$}
\begin{itemize}
\item{$\displaystyle\lim_{x \to \pm\infty}\frac{f(x)}{g(x)}=0$ if the degree
of $f(x)$ is less than the degree of $g(x)$.
\[ \text{Example:}\hspace{6 mm}\lim_{x \to \infty}\frac{x^2-2x}{x^3+3}=0 \]}
\item{$\displaystyle\lim_{x \to \pm\infty}\frac{f(x)}{g(x)}$ is infinite if the degree
of $f(x)$ is greater than the degree of $g(x)$.
\[ \text{Example:}\hspace{6 mm}\lim_{x \to \infty}\frac{x^3+2x}{x^2-8}=\infty \]}
\item{$\displaystyle\lim_{x \to \pm\infty}\frac{f(x)}{g(x)}$ is finite if the degree 
of $f(x)$ is equal to the degree of $g(x)$. The limit will be equal to the ratio of 
the leading coefficients of $f(x)$ to $g(x)$.
\[ \text{Example:}\hspace{6 mm}\lim_{x \to \infty}\frac{2x^2-3x+2}{10x-5x^2}=-\frac{2}{5} \]}
\end{itemize}

\subsubsection{Remarkable Limits}
See The Later Section.

\subsection{Intermediate Value Theorem}
A function $y=f(x)$ that is continuous on a closed interval $[a,b]$ takes on 
every value between $f(a)$ and $f(b)$.
As a more specific result, if $f$ is continuous on $[a,b]$ and $f(a)$ and $f(b)$ 
differ in sign, then the equation $f(x)=0$ has at least one solution 
in the open interval $[a,b]$.

\section{Derivatives}

\subsection{Notation of Derivatives}
The derivative of the function $y=f(x)$ is commonly written as any of the following:
\[ f'(x)\hspace{6 mm}[f(x)]'\hspace{6 mm}y'\]
\[ \diff{y}{x}\hspace{6 mm}\ddx f(x) \]

Higher order derivatives can be written in a number of ways. Below are the 
respective notations for second, fourth, and nth order derivatives in three different styles.
\[ \diff[2]{y}{x}\hspace{6 mm}\diff[4]{y}{x}\hspace{6 mm}\diff[n]{y}{x} \]
\[ f''(x)\hspace{6 mm}f^{IV}(x)\hspace{6 mm}f^{N}(x) \]
\[ f^{(2)}(x)\hspace{6 mm}f^{(4)}(x)\hspace{6 mm}f^{(n)}(x) \]

\subsection{Rate of Change}
If $(x_0,y_0)$ and $(x_1,y_1)$ are points on the graph of $y=f(x)$, then 
the \textbf{average rate of change} of $y$ with respect to $x$ over 
the interval $[x_0,x_1]$ is
\[ \frac{f(x_1)-f(x_0)}{x_1-x_0}=\frac{y_1-y_0}{x_1-x_0}=\frac{\Delta y}{\Delta x} \]

If $(x_0,y_0)$ is a point on the graph of $y=f(x)$, then the \textbf{instantaneous 
rate of change} of $y$ with respect to $x$ at $x_0$ is $f'(x_0)$.

\subsection{Definition of a Derivative}
The derivative of a function $f(x)$ at the point $x=a$ can be defined in either of two ways:
\[ f'(a)=\lim_{h \to 0}\frac{f(a+h)-f(a)}{h} \]
\[ f'(a)=\lim_{b \to a}\frac{f(b)-f(a)}{b-a} \]
If this limit exists at $a$, then $f(x)$ is said to be \textbf{differentiable} at $a$.

If a function is differentiable at a point $x=a$, it is continuous at that point.
The converse is false, i.e.\ continuity does not imply differentiability.

\subsection{Derivatives of Inverse Functions}\label{dinv}
If $f$ is differentiable at every point on an interval $I$, and $f'(x)\neq 0$
on $I$, then $g=f^{-1}(x)$ is differentiable at every point of the interior of
the interval $f(I)$ and $g'(f(x))=\frac{1}{f'(x)}$.

\subsection{Finding Maxima and Minima}
To find the maximum and minimum values of a function $y=f(x)$, locate
\begin{enumerate}
\item{the points where $f'(x)$ is zero or where $f'(x)$ fails to exist}
\item{the end points, if any, on the domain of $f(x)$}
\end{enumerate}
These are the only candidates for the value of $x$ where where $f(x)$ 
may have a maximum or a minimum.

\subsection{Monotonicity}
Let $f$ be differentiable for $a<x<b$ and continuous for $a\leq x\leq b$.
\begin{enumerate}
\item{If $f'(x)>0$ for every $x$ in $(a,b)$, then $f$ is increasing on $[a,b]$.}
\item{If $f'(x)<0$ for every $x$ in $(a,b)$, then $f$ is decreasing on $[a,b]$.}
\end{enumerate}

\subsection{Concavity}
Suppose that $f''(x)$ exists on the interval $(a,b)$.
\begin{enumerate}
\item{If $f''(x)>0$ in $(a,b)$, then $f$ is concave upward in $(a,b)$.}
\item{If $f''(x)<0$ in $(a,b)$, then $f$ is concave downward in $(a,b)$.}
\end{enumerate}
To locate the points of inflection of $y=f(x)$, find the points where 
$f''(x)=0$ or where $f''(x)$ fails to exist. These are the only candidates where 
$f(x)$ may have a point of inflection. Then test these points to make sure that 
$f''(x)<0$ on one side and $f''(x)>0$ on the other.

\subsection{Evaluating Derivatives}
For more details on evaluating derivatives of specific functions, see the later section.

The derivative of any constant function is $0$.
\[ \ddx a=0 \]
Constant coefficients can be factored out of derivatives.
\[ \ddx Af(x)=A\ddx f(x) \]
The derivative of a sum is the sum of the derivatives.
\[ \ddx [f(x)+g(x)]=\ddx f(x)+\ddx g(x) \]

\subsubsection{Power Rule}
\[ \ddx x^n=nx^{n-1} \]

\subsubsection{Product Rule}
\[ \ddx f\cdot g=f\cdot g'+g\cdot f' \]

\subsubsection{Quotient Rule}
\[ \ddx \frac{f}{g}=\frac{g\cdot f'-f\cdot g'}{g^2} \]

\subsubsection{Chain Rule}
The derivative of a composite function with respect to $x$ is equal to the derivative 
of the product of the derivative of the outer function with respect to the inner 
function and the derivative of the inner function.
\[ \ddx f(g(x))=f'(g(x))\cdot g'(x) \]

\subsubsection{Implicit Differentiation}
When an equation in $y$ and $x$ is not written in the form $y=f(x)$, we say the 
function is defined implicitly. To differentiate an implicit function, 
\begin{enumerate}
\item{From the given equation construct a function $F(x,y)=0$}
\item{Use the formula 
\[ \diff{y}{x}=-\frac{\diffp{F}{x}}{\diffp{F}{y}}=-\frac{F_x}{F_y} \]
In the numerator, differentiate $F$ with respect to $x$, treating $y$ as constant, 
and in the denominator, differentiate $F$ with respect to $y$, treating $x$ as constant.}
\end{enumerate}
Example: Find $\diff{y}{x}$ of $y=x^2+3\sin y$. 
\[ F(x,y)=y-x^2-3\sin y=0 \]
\begin{align*}
\diff{y}{x}=-\frac{F_x}{F_y} &= -\frac{-2x}{1-3\cos y}\\
    &= \frac{2x}{1-3\cos y}
\end{align*}

\subsection{Rolle's Theorem}
If $f$ is continuous on $[a,b]$ and differentiable on $(a.b)$ such that 
$f(a)=f(b)$, then there is at least one number $c$ in the open interval 
$(a,b)$ such that $f'(c)=0$.

Note that this is a special case of the Mean Value Theorem.

\subsection{Mean Value Theorem}
If $f$ is continuous on $[a,b]$ and differentiable on $(a,b)$, then there 
is at least one number $c$ in $(a,b)$ such that 
\[ \frac{f(b)-f(a)}{b-a}=f'(c)\Rightarrow f(b)-f(a)=f'(c)(b-a) \]

\subsection{Extreme Value Theorem}
If $f$ is continuous on $[a,b]$, then $f(x)$ has both a maximum and 
a minimum on $[a,b]$.

\subsection{Linear Approximation}
The linear approximation to $f(x)$ near $x=x_0$ is given by 
\[ y=f(x_0)+f'(x_0)(x-x_0) \]
for $x$ sufficiently close to $x_0$.

\subsection{Newton's Method}
Let $f$ be a differentiable function and suppose $r$ is a real zero of $f$.
If $x_n$ is an approximation to $r$, then the next approximation $x_{n+1}$ 
is given by
\[ x_{n+1}=x_n-\frac{f(x_n)}{f'(x_n)} \]
provided $f'(x_n)\neq 0$. Successive approximations can be found using this method.

\subsection{L'H\^{o}pital's Rule}
If $\displaystyle\lim_{x \to a}\frac{f(x)}{g(x)}$ is of the form $\frac{0}{0}$ 
or $\frac{\infty}{\infty}$, and if 
$\displaystyle\lim_{x \to a}\frac{f'(x)}{g'(x)}$ exists, then 
\[ \lim_{x \to a}\frac{f(x)}{g(x)}=\lim_{x \to a}\frac{f'(x)}{g'(x)} \]

\section{Integrals}

\subsection{Area Approximation Methods}
There are a few similar ways to approximate the area under a curve.
They all involve dividing the total area into smaller sections which approximate 
the shape of the enclosed region but which have easily computable areas.
The interval on the $x$-axis $[a,b]$ is typically divided into $n$ subintervals, 
each with length $\frac{b-a}{n}=\Delta x$.

\subsubsection{Left Riemann Sum}
\begin{align*}
\mathrm{Area} &\approx \Delta xf(a)+\Delta xf(a+\Delta x)+\Delta xf(a+2\Delta x)
+\ldots+\Delta xf(b-\Delta x)\\
    &\approx \Delta x[f(a)+f(a+\Delta x)+f(a+2\Delta x)+\ldots+f(b-\Delta x)]\\
    &\approx \Delta x \sum_{k=0}^{n-1} f(a+k\Delta x)
\end{align*}
This approximation will be an underestimation if $f$ is strictly increasing 
on $[a, b]$, and an overestimation if $f$ is strictly decreasing.

\subsubsection{Right Riemann Sum}
\begin{align*}
\mathrm{Area} &\approx \Delta xf(a+\Delta x)+\Delta xf(a+2\Delta x)
+\ldots+\Delta xf(b)\\
    &\approx \Delta x[f(a+\Delta x)+f(a+2\Delta x)+\ldots+f(b)]\\
    &\approx \Delta x \sum_{k=1}^{n} f(a+k\Delta x)
\end{align*}
This approximation will be an overestimation if $f$ is strictly increasing 
on $[a, b]$, and an underestimation if $f$ is strictly decreasing.

\subsubsection{Midpoint Rule}
\begin{align*}
\mathrm{Area} &\approx \Delta xf\left ( a+\frac{\Delta x}{2}\right )
+\Delta xf\left ( a+\frac{3\Delta x}{2}\right ) +\ldots
+\Delta xf\left ( b-\frac{\Delta x}{2}\right )\\
    &\approx \Delta x\left [ f\left ( a+\frac{\Delta x}{2}\right )
    +f\left ( a+\frac{3\Delta x}{2}\right )+\ldots
    +f\left ( b-\frac{\Delta x}{2}\right )\right ]\\
    &\approx \Delta x \sum_{k=1}^{n} f\left ( a+\frac{(2n-1)\Delta x}{2} \right )
\end{align*}

\subsubsection{Trapezoidal Rule}
\[ \mathrm{Area} \approx  \]
\subsection{Definition of the Definite Integral as the Limit of a Sum}

Lorem Ipsum

\end{document}
